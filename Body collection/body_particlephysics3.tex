\section{Lorentz\ boost}
Supposing\ the\ mass\ of\ electron\ is\ $m$,\ the\ total\ momentum\ is
\begin{align*}
    P^{\mu}&=(11.5,\sqrt{8^2-m^2}-\sqrt{3.5^2-m^2},0,0)\\
    &\simeq(11.5,4.5,0,0)
\end{align*}
where\ I\ approximated\ that\ electron\ mass\ is\ negligible\ because\ $m\sim0.5$\ MeV\ is\ much\ smaller\ than\ the\ energy\ scale\ GeV.\ And\ I\ represented\ above\ momentum\ with\ GeV.\ On\ the\ other\ hand\ in\ center\ of\ mass\ frame,\ supposing\ that\ the\ electron\ and\ the\ positron\ have\ opposite\ momentum\ $|\bm{p}|$,\ the\ total\ momentum\ is
\begin{align*}
    P^{\prime\mu}=(2\sqrt{\bm{p}^2+m^2},0,0,0)
\end{align*}
From\ the\ Lorentz\ invariant\ quantity
\begin{align*}
    4(\bm{p}^2+m^2)=11.5^2-4.5^2=112
\end{align*}
Therefore\ the\ Lorentz\ factor\ is
\begin{align*}
    P^0&=P^{\prime0}\gamma\\
    \gamma&=\frac{P^0}{P^{\prime0}}=\frac{11.5}{2\sqrt{\bm{p}^2+m^2}}=\frac{11.5}{\sqrt{112}}
\end{align*}
Therefore\ the\ Lorentz\ boost\ is
\begin{align}
    \beta\gamma&=\sqrt{1-\frac{1}{\gamma^2}}\gamma=\sqrt{\gamma^2-1}\nonumber\\
    &=\sqrt{\left(\frac{11.5}{\sqrt{112}}\right)^2-1}\simeq0.43
\end{align}
\clearpage
\section{Production\ of\ anti-proton}
From\ the\ baryon\ number\ conservation,\ I\ should\ consider\ the\ below\ process
\begin{align*}
    p\ +\ p\ \to\ p\ +\ p\ +\ p\ +\ \Bar{p}
\end{align*}
In\ the\ center\ of\ mass\ frame,\ all\ particles\ of\ final\ state\ are\ at\ rest\ for\ the\ minimum\ proton\ energy.\ The\ total\ momentum\ in\ laboratory\ frame\ is
\begin{align*}
    P^{\mu}=(E_{min}+m,|\bm{p}|,0,0)
\end{align*}
($\bm{p}$\ is\ the\ momentum\ of\ incoming\ proton\ and\ I\ work\ in\ natural\ unit)\ And\ the\ momentum\ in\ center\ of\ mass\ frame\ is
\begin{align*}
    P^{\prime\mu}=(4m,0,0,0)
\end{align*}
In\ order\ to\ relate\ these\ two\ momentum,\ I\ build\ a\ Lorentz\ invariant\ quantity.\ Therefore
\begin{align*}
    P^{\mu}P_{\mu}&=P^{\prime\mu}P_{\prime\mu}\\
    (E_{min}+m)^2-\bm{p}^2&=16m^2\\
    E_{min}^2+2mE_{min}-(E_{min}^2-m^2)&=15m^2\\
    2mE_{min}&=14m^2
\end{align*}
Therefore
\begin{align}
    E_{min}=4mc^2
\end{align}

\section{Class\ of\ meson}
Pseudo-scalar\ meson\ group
\begin{align}
    \pi^+\ \ ,\ \ \eta^{\prime}\ \ ,\ \ D_s^+
\end{align}
Vector\ meson\ group
\begin{align}
    \rho^+\ \ ,\ \ K_S^0\ \ ,\ \ \Upsilon
\end{align}

\section{Klein-Gordon\ eq}
\begin{align}
    \partial_{\mu}\frac{\delta\mathcal{L}}{\delta(\partial_{\mu}\phi)}-\frac{\delta\mathcal{L}}{\delta\phi}&=0\nonumber\\
    \partial_{\mu}\partial^{\mu}\phi-(-m^2\phi)&=0\nonumber\\
    (\partial_{\mu}\partial^{\mu}+m^2)\phi&=0
\end{align}

\section{Charge\ conjugation}
The\ solution\ of\ Dirac\ eq\ are
\begin{align*}
    \psi^{(1)}_p\propto e^{-ip\cdot x}\begin{pmatrix}
        1\\
        0\\
        \frac{p_z}{E+m}\\
        \frac{p_x+ip_y}{E+m}
    \end{pmatrix}\ \ &,\ \ \psi^{(2)}_p\propto e^{-ip\cdot x}\begin{pmatrix}
        0\\
        1\\
        \frac{p_x-ip_y}{E+m}\\
        \frac{-p_z}{E+m}
    \end{pmatrix}\\
    \psi^{(1)}_a\propto e^{ip\cdot x}\begin{pmatrix}
        \frac{p_x-ip_i}{E+m}\\
        \frac{-p_z}{E+m}\\
        0\\
        1
    \end{pmatrix}\ \ &,\ \ \psi^{(2)}_a\propto e^{ip\cdot x}\begin{pmatrix}
        \frac{p_z}{E+m}\\
        \frac{p_x+ip_y}{E+m}\\
        1\\
        0
    \end{pmatrix}
\end{align*}
The\ charge\ conjugation\ is\ the\ transformation\ which\ switches\ the\ particle\ and\ the\ anti-particle.\ So\ assuming\ 
\begin{align*}
    \Hat{C}\psi=i\gamma^2\psi^*
\end{align*}
I\ will\ confirm\ this\ transformation\ satisfies\ the\ above\ statement.
\begin{align}
    \Hat{C}\psi^{(1)}_p&\propto ie^{ip\cdot x}\begin{pmatrix}
        &&&-i\\
        &&i&\\
        &i&&\\
        -i&&&
    \end{pmatrix}\begin{pmatrix}
        1\\
        0\\
        \frac{p_z}{E+m}\\
        \frac{p_x-ip_y}{E+m}
    \end{pmatrix}\nonumber\\
    &=e^{ip\cdot x}\begin{pmatrix}
        &&&1\\
        &&-1&\\
        &-1&&\\
        1&&&
    \end{pmatrix}\begin{pmatrix}
        1\\
        0\\
        \frac{p_z}{E+m}\\
        \frac{p_x-ip_y}{E+m}
    \end{pmatrix}\nonumber\\
    &=e^{ip\cdot x}\begin{pmatrix}
        \frac{p_x-ip_y}{E+m}\\
        \frac{-p_z}{E+m}\\
        0\\
        1
    \end{pmatrix}=\psi^{(1)}_a
\end{align}

\begin{align}
    \Hat{C}\psi^{(2)}_p&\propto e^{ip\cdot x}\begin{pmatrix}
        &&&1\\
        &&-1&\\
        &-1&&\\
        1&&&
    \end{pmatrix}\begin{pmatrix}
        0\\
        1\\
        \frac{p_x+ip_y}{E+m}\\
        \frac{-p_z}{E+m}
    \end{pmatrix}\nonumber\\
    &=e^{ip\cdot x}\begin{pmatrix}
        \frac{-p_z}{E+m}\\
        \frac{-p_x-ip_y}{E+m}\\
        -1\\
        0
    \end{pmatrix}\propto e^{ip\cdot x}\begin{pmatrix}
        \frac{p_z}{E+m}\\
        \frac{p_x+ip_y}{E+m}\\
        1\\
        0
    \end{pmatrix}=\psi^{(2)}_a
\end{align}

\begin{align}
    \Hat{C}\psi^{(1)}_a&\propto e^{-ip\cdot x}\begin{pmatrix}
        &&&1\\
        &&-1&\\
        &-1&&\\
        1&&&
    \end{pmatrix}\begin{pmatrix}
        \frac{p_x+ip_i}{E+m}\\
        \frac{-p_z}{E+m}\\
        0\\
        1
    \end{pmatrix}\nonumber\\
    &=e^{-ip\cdot x}\begin{pmatrix}
        1\\
        0\\
        \frac{p_z}{E+m}\\
        \frac{p_x+ip_y}{E+m}
    \end{pmatrix}=\psi^{(1)}_p
\end{align}

\begin{align}
    \Hat{C}\psi^{(2)}_a&\propto e^{-ip\cdot x}\begin{pmatrix}
        &&&1\\
        &&-1&\\
        &-1&&\\
        1&&&
    \end{pmatrix}\begin{pmatrix}
        \frac{p_z}{E+m}\\
        \frac{p_x-ip_y}{E+m}\\
        1\\
        0
    \end{pmatrix}\nonumber\\
    &=e^{-ip\cdot x}\begin{pmatrix}
        0\\
        -1\\
        \frac{-p_x+ip_y}{E+m}\\
        \frac{p_z}{E+m}
    \end{pmatrix}\propto e^{-ip\cdot x}\begin{pmatrix}
        0\\
        1\\
        \frac{p_x-ip_y}{E+m}\\
        \frac{-p_z}{E+m}
    \end{pmatrix}=\psi^{(2)}_p
\end{align}
Therefore\ the\ given\ formula\ of\ charge\ conjugation\ switches\ the\ particle\ and\ the\ anti-particle.
\clearpage
\section{Branching\ fractions\ of\ Higgs}
The\ coupling\ constant\ between\ Higgs\ and\ fermions\ is\ proportional\ to\ fermion\ mass.\ So\ the\ fraction\ of\ branching\ fractions\ is\ equal to\ square\ of\ the\ fraction\ of\ masses.\ Therefore
\begin{align}
    \frac{B_{b\Bar{b}}}{B_{\tau\tau}}=\left(\frac{m_b}{m_{\tau}}\right)^2\simeq\left(\frac{4.2\times10^3}{1.8\times10^3}\right)^2\simeq5
\end{align}

\begin{align}
    \frac{B_{c\Bar{c}}}{B_{\tau\tau}}=\left(\frac{m_c}{m_{\tau}}\right)^2\simeq\left(\frac{1.2\times10^3}{1.8\times10^3}\right)^2\simeq0.4
\end{align}

\section{Feynmann\ diagram}
I\ drawed\ on\ next\ page.

\section{Branching\ fraction\ of\ B}
From\ the\ problem\ statement,\ I\ consider\ the\ decay\ processes\ via\ a\ weak\ interaction.\ The\ decay\ of\ B\ meson\ is\ mainly\ contributed\ by\ the\ transition\ of\ bottom\ quark.

Because\ the\ mass\ of\ bottom\ quark\ is\ 4\ GeV\ and\ the\ mass\ of\ top\ quark\ is\ 170\ GeV,\ the\ transition\ to\ top\ quark\ is\ negligible.\ And\ because
\begin{align*}
    V_{ub}\simeq4\times10^{-3}\ \ ,\ \ V_{cb}\simeq4\times10^{-2}
\end{align*}
the\ branching\ fraction\ of\ the\ transition\ of\ bottom\ quark\ is
\begin{align*}
    \frac{Br(b\to u)}{Br(b\to c)}=\left(\frac{V_{ub}}{V_{cb}}\right)^2\simeq10^{-2}
\end{align*}
suppressed.\ So\ the\ transition\ $b\to u$\ is\ negligible.

Therefore\ I\ should\ take\ into\ account\ the\ transition\ $b\to s$\ with\ loop\ diagram.

I\ can't\ estimate\ branching\ fraction\ qualitatively.

\section{Detector\ model}
\subsection{Probability\ density\ function}
Because\ the\ only\ $\Delta L$\ in\ range\ of\ 
\begin{align*}
    -\frac{l}{2}\leq\Delta L\leq\frac{l}{2}
\end{align*}
is\ allowed,\ the\ probability\ function\ is
\begin{align}
    P(\Delta L)=\frac{1}{l}\theta(l/2-|\Delta L|)
\end{align}
\subsection{The\ standard\ deviation}
The\ mean\ value\ is
\begin{align*}
    \mu&=\int^{\infty}_{-\infty}xP(x)dx\\
    &=\frac{1}{l}\int^{\infty}_{-\infty}x\theta(l/2-|x|)\\
    &=\frac{1}{l}\int^{\frac{l}{2}}_{-\frac{l}{2}}xdx=\frac{1}{l}\left[\frac{x^2}{2}\right]^{\frac{l}{2}}_{-\frac{l}{2}}=0
\end{align*}
So\ the\ variance\ is
\begin{align*}
    \sigma^2&=\int^{\infty}_{-\infty}x^2P(x)dx\\
    &=\frac{1}{l}\left[\frac{x^3}{3}\right]^{\frac{l}{2}}_{-\frac{l}{2}}=\frac{l^2}{12}
\end{align*}
Therefore\ the\ standard\ deviation\ is
\begin{align}
    \sigma=\sqrt{\frac{l^2}{12}}=\frac{l}{2\sqrt{3}}
\end{align}

\section{The\ experimental\ set\ up\ of\ Daya\ Bay\ experiment}
\textbf{Abbreviations\ and\ terminology}

At\ first,\ I\ will\ explain\ the\ some\ abbreviations\ and\ terminology.\ "Gd\ LS"\ is\ the\ gadolinium-doped\ liquid\ scintillator.

"ADs"\ means\ anti-neutrino\ detectors.\ This\ is\ 3\ nested\ cylindrical\ vessels.\ The\ inner\ acrylic\ vessel\ is\ filled\ with\ 0.1\ \%\ GD\ LS\ which\ is\ 20t.\ This\ is\ the\ anti-neutrino\ target.\ On\ the\ other\ hand,\ the\ outer\ acrylic\ vessel\ is\ filled\ with\ the\ undoped\ LS.\ This\ increases\ the\ efficiency\ for\ detecting\ photon.

"EH"\ is\ the\ experimental\ hall,\ in\ which\ ADs\ is\ installed.\\
\textbf{The\ set\ up}

"PMTs"\ means\ photomultiplier\ tubes.

There\ are\ three\ EHs\ (EH1,\ EH2,\ EH3).\ EH1\ and\ EH2\ include\ two\ ADs\ and\ EH3\ includes\ four\ ADs.\ The\ position\ relationship\ in\ EHs\ is\ that\ EH1\ is\ near\ EH2,\ and\ EH3\ is\ far\ from\ these\ two\ EHs.\ 

And\ there\ are\ six\ reactor\ core\ relatively\ near\ EH1\ and\ EH2,\ in\ which\ anti-neutrino\ is\ produced.\ Two\ of\ cores\ are\ 365m\ from\ EH1,\ and\ four\ of\ cores\ are\ 505m\ from\ EH2.\ And\ the\ average\ distance\ to\ EH3\ is\ 1663m.

And\ PMTs\ are\ uniformly\ positioned\ on the\ cylinder\ of\ outermost\ stainless\ steel\ vessel\ and\ immersed\ in\ mineral\ oil.\\
\textbf{History}

This\ experiment\ started\ to\ take\ the\ data\ with\ six\ ADs\ on\ 2011/12/24.\ A\ AD\ was\ installed\ in\ EH2\ on\ 2012/7/28\ and\ the\ last\ AD\ was\ installed\ in\ EH3\ on\ 2012/10/19.\ And\ this\ experiment\ lasted\ until\ 2016/12/20.

\section{The\ D0\ collaboration}
The\ signature\ is\ that\ top\ quark\ is\ produced\ in\ pair\ by\ strong\ interaction\ and\ decay\ according\ to\ $t\Bar{t}\to W^+W^-b\Bar{b}$.\ The\ reason\ why\ top\ quark\ don't\ form\ a\ hadron\ is\ that\ the\ lifetime\ is\ too\ short.\ In\ particular,\ the\ weak\ bosons\ have\ the\ decaying\ channels\ that\ are\ dilepton\ channel\ and\ single-lepton\ channel.\ In\ the\ dilepton\ channel\ the\ two\ weak\ bosons\ decay\ to\ $e\mu$+jets,\ ee+jets,\ $\mu\mu$+jets.\ In\ the\ single-lepton\ channel\ a\ weak\ boson\ decays\ to\ $e$+jets,\ $\mu$+jets.

Next,\ I\ will\ explain\ the\ detector\ setup\ to\ catch\ this\ signature.\ The\ detector\ surrounds\ the\ $p\Bar{p}$\ collision\ point\ and\ the\ detector\ consists\ of\ three\ nested\ shells.\ The\ innermost\ shell\ is\ the\ tracking\ and\ transition\ radiation\ detectors,\ which\ consists\ of\ the\ Central\ detector,\ the\ Vertex\ Drift\ Chamber,\ the\ transition\ radiation\ detector,\ the\ Central\ Drift\ Chamber,\ two\ Forward\ Drift\ Chambers.\ And\ the\ role\ of\ this\ shell\ is\ distinguish\ single\ electrons\ from\ clese-spaced\ convention\ pairs.The\ middle\ shell\ is\ calorimetry,\ which\ measures\ the\ energy\ of\ the\ electrons,\ photons,\ jets\ and\ identifies\ them.\ The\ outermost\ shell\ is\ the\ muon\ detectors,\ which\ identifies\ produced\ muons\ and\ determines\ their\ trajectories\ and\ momenta.