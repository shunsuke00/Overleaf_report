\section{Explaining\ the\ mass\ dependence}
The\ differential\ event\ rate\ is
\begin{align*}
    \frac{dR}{dE_R}\propto\frac{1}{m_{DM}}\frac{d\sigma}{dE_R}\int^{v_{esc}}_{v_{min}}dv_{DM}\frac{f(v_{DM})}{v_{DM}}
\end{align*}
Therefore\ the\ number\ of\ events\ is
\begin{align*}
    R\propto\frac{\sigma}{m_{DM}}\int^{v_{esc}}_{v_{min}}dv_{DM}\frac{f(v_{DM})}{v_{DM}}
\end{align*}
where
\begin{align*}
    v_{min}=\sqrt{\frac{m_NE_R}{2\mu^2}}
\end{align*}
When\ the\ mass\ of\ dark\ matter\ is\ much\ larger\ than\ 100\ GeV
\begin{align*}
    \mu=\frac{m_{DM}m_N}{m_{DM}+m_N}\simeq m_N
\end{align*}
the\ integration\ don't\ depend\ on\ the\ mass\ of\ dark\ matter.\ Because\ the\ number\ of\ events\ is\ observable,\ the\ cross\ section\ is\ proportional\ to\ the\ mass\ of\ dark\ matter.

\section{Explaining\ the\ weaker\ constraint}
When\ $m_{DM}\ll$30\ GeV, $\mu$ decreases\ steeply as the mass of dark matter decreases. Then $v_{min}$ becomes larger, so the range of integration becomes narrower. Therefore the value of integration becomes smaller and the cross section increases.